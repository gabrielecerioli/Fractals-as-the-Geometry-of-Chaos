\textit{A black swan is an event in the stock market that exceeds six standard deviations, a nearly impossible event. }
\begin{itemize}
    \item On October 19, 1987 (Black Monday), the Dow Jones Industrial Average fell 29.2 percent, based on classical theories the probability of such an event was less than one in $10^{50}$ odds. This is a number outside the scale of nature, it is like tossing 166 coins and getting 166 heads. Imagine choosing one specific molecule of water from all the oceans on Earth — that’s roughly a 1 in $10^{46}$ chance. Now make it 10000 times rarer.
\end{itemize}
It surely had been an event of extremely bad luck, a freak accident, an unpredictable "act of God", or was it?
\begin{itemize}

\item In the summer of 1998 the improbable happened: on August 4 the Dow index fell 3.5$\%$, three weeks later by 4.4$\%$ and then again on August 31 by 6.8$\%$. This should never have happened, about one in 500 billion chance. You should not expect to see something like this even if you traded for 2.5 million years.
\item A year earlier (1997), the Dow had fallen 7.7$\%$ in one day with a probability of one in 50 billion.
\item In July 2002, the index recorded three steep falls within seven trading days, (one in four trillion).
\end{itemize}
\textbf{The reader should already have grasped, without the need to list more of the many such events, that there is something fundamentally flawed in the way the probability of these rare events is assessed.}